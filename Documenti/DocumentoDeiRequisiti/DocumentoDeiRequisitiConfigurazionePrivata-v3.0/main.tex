\documentclass[a4paper,11pt]{article}       % Articolo, Foglio A4, Carattere 11
\usepackage[T1]{fontenc}                    % Codifica dei font
\usepackage[utf8]{inputenc}                 % Codifica dell'input da tastiera
\usepackage[italian]{babel}                 % Lingua: italiano
\usepackage{hyperref}                       % Collegamenti ipertestuali
\usepackage[toc]{glossaries}                % Gestione glossario


% Set-up url
\hypersetup
{
    colorlinks=true,
    linkcolor=black,
    urlcolor=blue,
}

% Dichiarazione glossario
\makeglossaries

\newglossaryentry{Human-centered design}
{
    name = Human-centered design,
    description = {approccio di problem solving che coinvolge la prospettiva del cliente in tutti gli step della risoluzione stessa}
}

\newglossaryentry{Two-way mirror}
{
    name = Two-way mirror,
    description = {particolare tipo di specchio che da un lato riflette la luce mentre dall'altro ne permette il passaggio}
}

\begin{document}
\begin{titlepage}
    \centering
        \vspace*{1in}
        \begin{Large}
            MagicMirror-GBM 
            
            configurazione ad uso privato\par
        \end{Large} 
        \vspace{1.5in}
        \vfill
        Documento dei requisiti\\
        \vspace{0.1in}ver. 3.0
\end{titlepage}

% numerazione romana delle pagine
\pagenumbering{roman}

%Indice dei contenuti
\tableofcontents
\newpage

%Numerazione araba delle pagine
\pagenumbering{arabic}

\section{Premesse del progetto}
    \subsection{Obiettivi e scopo del progetto}
        Piattaforma Open Source modulare per trasformare un classico specchio in un sistema digitale multifunzione.
    \subsection{Contesto di business}
        Nella continua evoluzione tecnologica degli ultimi anni si è rilevata sempre più utile l'integrazione della domotica e della tecnologia in strumenti di uso quotidiano.
    \subsection{Stakeholders}
        Le figure che influenzano lo sviluppo del sistema software sono:
        \begin{itemize}
            \item Committente: \textbf{NonSoloTelefonia Lab}
            \item Clienti: \textbf{\gls{Human-centered design}}
            \item Developers (analisti, progettisti)
        \end{itemize}

\newpage

\section{Servizi del sistema}
    
    \subsection{Requisiti funzionali}
        \begin{itemize}
            \item[2.1.1] Il sistema dovrà consentire la modifica delle impostazioni del sistema software stesso.
                \begin{itemize}
                    \item[2.1.1.1] Il sistema dovrà consentire la gestione delle impostazioni di connettività.
                    \item[2.1.1.2] Il sistema dovrà consentire la gestione del microfono e della fotocamera.
                    \item[2.1.1.3] Il sistema dovrà permettere il reset del sistema stesso.
                    \item[2.1.1.4] Il sistema dovrà permettere una fase di configurazione iniziale.
                    \item[2.1.1.5] Il sistema dovrà permettere la modifica della lingua.
                    \item[2.1.1.6] Il sistema dovrà permettere la regolazione del volume.
                    \item[2.1.1.7] Il sistema dovrà permettere la modifica delle suonerie.
                    \item[2.1.1.8] Il sistema dovrà permettere la gestione delle notifiche.
                    \item[2.1.1.9] Il sistema dovrà consentire la modifica dello sfondo.
                \end{itemize}
            \item[2.1.2] Il sistema dovrà integrare il modulo \textbf{MMM-AirQuality}.
            \item[2.1.3] Il sistema dovrà integrare il modulo \textbf{MMM-DHT}.
            \item[2.1.4] Il sistema dovrà integrare il modulo \textbf{newsfeed}.
            \item[2.1.5] Il sistema dovrà integrare il modulo \textbf{MMM-AVStock}.
            \item[2.1.6] Il sistema dovrà integrare il modulo \textbf{clock}.
            \item[2.1.7] Il sistema dovrà integrare il modulo \textbf{weather}.
            \item[2.1.8] Il sistema dovrà integrare il modulo \textbf{weatherforecast}.
            \item[2.1.9] Il sistema dovrà integrare il modulo \textbf{calendar}.
            \item[2.1.10] Il sistema dovrà integrare il modulo \textbf{alert}.
            \item[2.1.11] Il sistema dovrà integrare il modulo \textbf{MMM-Memo}.
            \item[2.1.12] Il sistema dovrà integrare il modulo \textbf{MMM-Screencast}.
            \item[2.1.13] Il sistema dovrà integrare il modulo \textbf{MMM-Mail}.
            \item[2.1.14] Il sistema dovrà integrare il modulo \textbf{raspotify}.
            \item[2.1.15] Il sistema dovrà integrare il modulo \textbf{MMM-Online-State}.
        \end{itemize}
        
    \subsection{Requisiti informativi}
        Il linguaggio utilizzato per lo scambio di informazioni tra le componenti interne ed esterne del sistema è il JSON.

\newpage

\section{Vincoli di sistema}
    
    \subsection{Requisiti di interfaccia}
        L'interfaccia proposta dal sistema è stata appositamente studiata per garantire una fruizione di contenuti intuitiva ed immediata.
        
        \begin{itemize}
            \item[3.1.1] Interfaccia principale
                \begin{itemize}
                    \item[3.1.1.1] Visualizzazione della qualità dell'aria per la zona specificata.
                    \item[3.1.1.2] Visualizzazione della temperatura e dell'umidità locali (usa sensore: DHT11).
                    \item[3.1.1.3] Visualizzazione delle news più recenti.
                    \item[3.1.1.4] Visualizzazione degli aggiornamenti relativi alle quotazioni in borsa.
                    \item[3.1.1.5] Visualizzazione della data e dell'ora correnti.
                    \item[3.1.1.6] Visualizzazione delle previsioni meteo.
                    \item[3.1.1.7] Visualizzazione di un calendario interattivo.
                    \item[3.1.1.8] Visualizzazione delle annotazioni.
                    \item[3.1.1.9] Visualizzazione delle email in entrata.
                    \item[3.1.1.10] Visualizzazione dello stato della connessione.
                \end{itemize}   
                
            \item[3.1.2] Interfaccia modulo MMM-Screencast
                \begin{itemize}
                    \item[3.1.2.1] Visualizzazione di contenuti multimediali.
                \end{itemize}
        \end{itemize}
    
    \subsection{Requisiti tecnologici}
        L'intero progetto è stato realizzato utilizzando i seguenti strumenti:
        \begin{itemize}
            \item Raspberry Pi modello 2 o superiore
            \item Sensore DHT11
            \item Casse audio
            \item Schermo con interfaccia HDMI
            \item Telaio specchio
            \item \gls{Two-way mirror}
        \end{itemize}
        
    \subsection{Requisiti di prestazione}
        Non si registrano particolari esigenze in questo ambito.
    
    \subsection{Requisiti di sicurezza}
        Non si registrano particolari esigenze in questo ambito.
    
    \subsection{Requisiti operativi}
        L'intero progetto è stato realizzato utilizzando i seguenti linguaggi:
        \begin{itemize}
            \item JavaScript
            \item CSS
            \item HTML
            \item PHP
        \end{itemize}
        L'intero progetto è basato sulle seguenti piattaforme:
        \begin{itemize}
            \item npm + Node.js v10.x o superiore
            \item Electron
        \end{itemize}
        Si relaziona con sistemi operativi Raspberry Pi OS (full version).
    
    \subsection{Requisiti politici e legali}
        Il sistema software open source è rilasicato sotto la licenza \href{https://github.com/AndreaGrandieri/MagicMirror-GBM/blob/main/LICENSE}{Apache-2.0}.
    
    \subsection{Vincoli API esterne}
        L'utilizzo di API esterne è soggetto a limitazioni poste dai fornitori delle API stesse. Pertanto si invita ad una consultazione dei regolamenti di utilizzo delle singole API.
\clearpage
    \printglossary[nonumberlist]

\end{document}
